\documentclass[]{article}
\usepackage{lmodern}
\usepackage{amssymb,amsmath}
\usepackage{ifxetex,ifluatex}
\usepackage{fixltx2e} % provides \textsubscript
\ifnum 0\ifxetex 1\fi\ifluatex 1\fi=0 % if pdftex
  \usepackage[T1]{fontenc}
  \usepackage[utf8]{inputenc}
\else % if luatex or xelatex
  \ifxetex
    \usepackage{mathspec}
  \else
    \usepackage{fontspec}
  \fi
  \defaultfontfeatures{Ligatures=TeX,Scale=MatchLowercase}
\fi
% use upquote if available, for straight quotes in verbatim environments
\IfFileExists{upquote.sty}{\usepackage{upquote}}{}
% use microtype if available
\IfFileExists{microtype.sty}{%
\usepackage{microtype}
\UseMicrotypeSet[protrusion]{basicmath} % disable protrusion for tt fonts
}{}
\usepackage[margin=1in]{geometry}
\usepackage{hyperref}
\hypersetup{unicode=true,
            pdftitle={Estimating trait means and covariances through multiple imputation},
            pdfauthor={Alexey N. Shiklomanov, Elizabeth M. Cowdery, Michael Bahn, Chaeho Byun, Joseph Craine, Andrés Gonzalez-Melo, Steven Jansen, Nathan Kraft, Koen Kramer, Vanessa Minden, Ülo Niinemets, Yusuke Onoda, Enio Egon Sosinski, Nadejda A. Soudzilovskaia, Michael C. Dietze},
            pdfborder={0 0 0},
            breaklinks=true}
\urlstyle{same}  % don't use monospace font for urls
\usepackage{color}
\usepackage{fancyvrb}
\newcommand{\VerbBar}{|}
\newcommand{\VERB}{\Verb[commandchars=\\\{\}]}
\DefineVerbatimEnvironment{Highlighting}{Verbatim}{commandchars=\\\{\}}
% Add ',fontsize=\small' for more characters per line
\usepackage{framed}
\definecolor{shadecolor}{RGB}{248,248,248}
\newenvironment{Shaded}{\begin{snugshade}}{\end{snugshade}}
\newcommand{\AlertTok}[1]{\textcolor[rgb]{0.94,0.16,0.16}{#1}}
\newcommand{\AnnotationTok}[1]{\textcolor[rgb]{0.56,0.35,0.01}{\textbf{\textit{#1}}}}
\newcommand{\AttributeTok}[1]{\textcolor[rgb]{0.77,0.63,0.00}{#1}}
\newcommand{\BaseNTok}[1]{\textcolor[rgb]{0.00,0.00,0.81}{#1}}
\newcommand{\BuiltInTok}[1]{#1}
\newcommand{\CharTok}[1]{\textcolor[rgb]{0.31,0.60,0.02}{#1}}
\newcommand{\CommentTok}[1]{\textcolor[rgb]{0.56,0.35,0.01}{\textit{#1}}}
\newcommand{\CommentVarTok}[1]{\textcolor[rgb]{0.56,0.35,0.01}{\textbf{\textit{#1}}}}
\newcommand{\ConstantTok}[1]{\textcolor[rgb]{0.00,0.00,0.00}{#1}}
\newcommand{\ControlFlowTok}[1]{\textcolor[rgb]{0.13,0.29,0.53}{\textbf{#1}}}
\newcommand{\DataTypeTok}[1]{\textcolor[rgb]{0.13,0.29,0.53}{#1}}
\newcommand{\DecValTok}[1]{\textcolor[rgb]{0.00,0.00,0.81}{#1}}
\newcommand{\DocumentationTok}[1]{\textcolor[rgb]{0.56,0.35,0.01}{\textbf{\textit{#1}}}}
\newcommand{\ErrorTok}[1]{\textcolor[rgb]{0.64,0.00,0.00}{\textbf{#1}}}
\newcommand{\ExtensionTok}[1]{#1}
\newcommand{\FloatTok}[1]{\textcolor[rgb]{0.00,0.00,0.81}{#1}}
\newcommand{\FunctionTok}[1]{\textcolor[rgb]{0.00,0.00,0.00}{#1}}
\newcommand{\ImportTok}[1]{#1}
\newcommand{\InformationTok}[1]{\textcolor[rgb]{0.56,0.35,0.01}{\textbf{\textit{#1}}}}
\newcommand{\KeywordTok}[1]{\textcolor[rgb]{0.13,0.29,0.53}{\textbf{#1}}}
\newcommand{\NormalTok}[1]{#1}
\newcommand{\OperatorTok}[1]{\textcolor[rgb]{0.81,0.36,0.00}{\textbf{#1}}}
\newcommand{\OtherTok}[1]{\textcolor[rgb]{0.56,0.35,0.01}{#1}}
\newcommand{\PreprocessorTok}[1]{\textcolor[rgb]{0.56,0.35,0.01}{\textit{#1}}}
\newcommand{\RegionMarkerTok}[1]{#1}
\newcommand{\SpecialCharTok}[1]{\textcolor[rgb]{0.00,0.00,0.00}{#1}}
\newcommand{\SpecialStringTok}[1]{\textcolor[rgb]{0.31,0.60,0.02}{#1}}
\newcommand{\StringTok}[1]{\textcolor[rgb]{0.31,0.60,0.02}{#1}}
\newcommand{\VariableTok}[1]{\textcolor[rgb]{0.00,0.00,0.00}{#1}}
\newcommand{\VerbatimStringTok}[1]{\textcolor[rgb]{0.31,0.60,0.02}{#1}}
\newcommand{\WarningTok}[1]{\textcolor[rgb]{0.56,0.35,0.01}{\textbf{\textit{#1}}}}
\usepackage{graphicx,grffile}
\makeatletter
\def\maxwidth{\ifdim\Gin@nat@width>\linewidth\linewidth\else\Gin@nat@width\fi}
\def\maxheight{\ifdim\Gin@nat@height>\textheight\textheight\else\Gin@nat@height\fi}
\makeatother
% Scale images if necessary, so that they will not overflow the page
% margins by default, and it is still possible to overwrite the defaults
% using explicit options in \includegraphics[width, height, ...]{}
\setkeys{Gin}{width=\maxwidth,height=\maxheight,keepaspectratio}
\IfFileExists{parskip.sty}{%
\usepackage{parskip}
}{% else
\setlength{\parindent}{0pt}
\setlength{\parskip}{6pt plus 2pt minus 1pt}
}
\setlength{\emergencystretch}{3em}  % prevent overfull lines
\providecommand{\tightlist}{%
  \setlength{\itemsep}{0pt}\setlength{\parskip}{0pt}}
\setcounter{secnumdepth}{0}
% Redefines (sub)paragraphs to behave more like sections
\ifx\paragraph\undefined\else
\let\oldparagraph\paragraph
\renewcommand{\paragraph}[1]{\oldparagraph{#1}\mbox{}}
\fi
\ifx\subparagraph\undefined\else
\let\oldsubparagraph\subparagraph
\renewcommand{\subparagraph}[1]{\oldsubparagraph{#1}\mbox{}}
\fi

%%% Use protect on footnotes to avoid problems with footnotes in titles
\let\rmarkdownfootnote\footnote%
\def\footnote{\protect\rmarkdownfootnote}

%%% Change title format to be more compact
\usepackage{titling}

% Create subtitle command for use in maketitle
\newcommand{\subtitle}[1]{
  \posttitle{
    \begin{center}\large#1\end{center}
    }
}

\setlength{\droptitle}{-2em}

  \title{Estimating trait means and covariances through multiple imputation}
    \pretitle{\vspace{\droptitle}\centering\huge}
  \posttitle{\par}
    \author{Alexey N. Shiklomanov, Elizabeth M. Cowdery, Michael Bahn, Chaeho Byun,
Joseph Craine, Andrés Gonzalez-Melo, Steven Jansen, Nathan Kraft, Koen
Kramer, Vanessa Minden, Ülo Niinemets, Yusuke Onoda, Enio Egon Sosinski,
Nadejda A. Soudzilovskaia, Michael C. Dietze}
    \preauthor{\centering\large\emph}
  \postauthor{\par}
    \date{}
    \predate{}\postdate{}
  

\begin{document}
\maketitle

\hypertarget{introduction}{%
\section{Introduction}\label{introduction}}

Under \emph{single} imputation, the imputation step occurs once outside
of the MCMC loop. In \emph{multiple} imputation, as implemented in this
paper, the imputation step is part of the MCMC loop, such that imputed
data values are conditioned on the current state of the parameters and
vice-versa. In the figure below, \(Y\) is the original data (with
missing values), \(Y^*\) is the original data with missing values
imputed, \(\mu_p\) and \(\Sigma_p\) are values of the mean vector and
covariance matrix (respectively) calculated only from the original data,
and \(\mu^*_t\) and \(\Sigma^*_t\) are the draws of the mean vector and
covariance matrix (respectively) at MCMC iteration \(t\).

\includegraphics{multiple_imputation_files/figure-latex/diagram-1.pdf}

\hypertarget{demonstration}{%
\section{Demonstration}\label{demonstration}}

Consider two positively correlated traits, A and B.

\begin{Shaded}
\begin{Highlighting}[]
\KeywordTok{set.seed}\NormalTok{(}\DecValTok{12345}\NormalTok{) }\CommentTok{# For reproducibility}
\KeywordTok{library}\NormalTok{(mvtraits)}

\NormalTok{true_mu <-}\StringTok{ }\KeywordTok{c}\NormalTok{(}\DecValTok{0}\NormalTok{, }\DecValTok{0}\NormalTok{)}
\NormalTok{true_cov <-}\StringTok{ }\KeywordTok{matrix}\NormalTok{(}\KeywordTok{c}\NormalTok{(}\DecValTok{1}\NormalTok{, }\FloatTok{0.8}\NormalTok{, }\FloatTok{0.8}\NormalTok{, }\DecValTok{1}\NormalTok{), }\DataTypeTok{nrow =} \DecValTok{2}\NormalTok{)}
\NormalTok{Y_all <-}\StringTok{ }\KeywordTok{random_mvnorm}\NormalTok{(}\DecValTok{1000}\NormalTok{, true_mu, true_cov)}
\KeywordTok{colnames}\NormalTok{(Y_all) <-}\StringTok{ }\KeywordTok{c}\NormalTok{(}\StringTok{"A"}\NormalTok{, }\StringTok{"B"}\NormalTok{)}
\KeywordTok{plot}\NormalTok{(Y_all, }\DataTypeTok{pch =} \DecValTok{19}\NormalTok{)}
\end{Highlighting}
\end{Shaded}

\includegraphics{multiple_imputation_files/figure-latex/create_traits-1.pdf}

Simulate missingness by randomly removing half of the data.

\begin{Shaded}
\begin{Highlighting}[]
\NormalTok{Y <-}\StringTok{ }\NormalTok{Y_all}
\NormalTok{Y[}\KeywordTok{sample.int}\NormalTok{(}\DecValTok{2000}\NormalTok{, }\DecValTok{1000}\NormalTok{)] <-}\StringTok{ }\OtherTok{NA}
\KeywordTok{plot}\NormalTok{(Y_all, }\DataTypeTok{pch =} \DecValTok{19}\NormalTok{, }\DataTypeTok{col =} \StringTok{"grey"}\NormalTok{)}
\KeywordTok{points}\NormalTok{(Y, }\DataTypeTok{pch =} \DecValTok{19}\NormalTok{, }\DataTypeTok{col =} \StringTok{"black"}\NormalTok{)}
\KeywordTok{legend}\NormalTok{(}
  \StringTok{"topleft"}\NormalTok{,}
  \DataTypeTok{legend =} \KeywordTok{c}\NormalTok{(}\StringTok{"True missing"}\NormalTok{, }\StringTok{"Present"}\NormalTok{),}
  \DataTypeTok{col =} \KeywordTok{c}\NormalTok{(}\StringTok{"grey"}\NormalTok{, }\StringTok{"black"}\NormalTok{),}
  \DataTypeTok{pch =} \DecValTok{19}\NormalTok{,}
  \DataTypeTok{bg =} \StringTok{"white"}
\NormalTok{)}
\end{Highlighting}
\end{Shaded}

\includegraphics{multiple_imputation_files/figure-latex/remove_na-1.pdf}

Set an uninformative multivariate normal prior on \(\mu\)\ldots{}

\begin{Shaded}
\begin{Highlighting}[]
\NormalTok{mu_}\DecValTok{0}\NormalTok{ <-}\StringTok{ }\KeywordTok{c}\NormalTok{(}\DecValTok{0}\NormalTok{, }\DecValTok{0}\NormalTok{)}
\NormalTok{Sigma_}\DecValTok{0}\NormalTok{ <-}\StringTok{ }\KeywordTok{diag}\NormalTok{(}\DecValTok{10}\NormalTok{, }\DecValTok{2}\NormalTok{)}
\end{Highlighting}
\end{Shaded}

\ldots{}and an uninformative Wishart prior on \(\Sigma\).

\begin{Shaded}
\begin{Highlighting}[]
\NormalTok{v0 <-}\StringTok{ }\DecValTok{2}
\NormalTok{S0 <-}\StringTok{ }\KeywordTok{diag}\NormalTok{(}\DecValTok{10}\NormalTok{, }\DecValTok{2}\NormalTok{)}
\end{Highlighting}
\end{Shaded}

Take an initial guess at the mean and covariance by drawing from their
priors.

\begin{Shaded}
\begin{Highlighting}[]
\NormalTok{mu_star <-}\StringTok{ }\KeywordTok{random_mvnorm}\NormalTok{(}\DecValTok{1}\NormalTok{, mu_}\DecValTok{0}\NormalTok{, Sigma_}\DecValTok{0}\NormalTok{)[}\DecValTok{1}\NormalTok{,]}
\NormalTok{mu_star}
\end{Highlighting}
\end{Shaded}

\begin{verbatim}
## [1] -7.3409439  0.7029685
\end{verbatim}

\begin{Shaded}
\begin{Highlighting}[]
\NormalTok{Sigma_star <-}\StringTok{ }\KeywordTok{rWishart}\NormalTok{(}\DecValTok{1}\NormalTok{, v0, S0)[,,}\DecValTok{1}\NormalTok{]}
\NormalTok{Sigma_star}
\end{Highlighting}
\end{Shaded}

\begin{verbatim}
##          [,1]      [,2]
## [1,] 81.17255 15.526650
## [2,] 15.52665  4.779821
\end{verbatim}

Impute values based on this initial guess.

\begin{Shaded}
\begin{Highlighting}[]
\NormalTok{Ystar <-}\StringTok{ }\KeywordTok{mvnorm_fill_missing}\NormalTok{(Y, mu_star, Sigma_star)}

\NormalTok{mylegend <-}\StringTok{ }\ControlFlowTok{function}\NormalTok{() \{}
  \KeywordTok{legend}\NormalTok{(}
    \StringTok{"topleft"}\NormalTok{,}
    \DataTypeTok{legend =} \KeywordTok{c}\NormalTok{(}\StringTok{"True missing"}\NormalTok{, }\StringTok{"Present"}\NormalTok{, }\StringTok{"Imputed"}\NormalTok{),}
    \DataTypeTok{col =} \KeywordTok{c}\NormalTok{(}\StringTok{"grey"}\NormalTok{, }\StringTok{"black"}\NormalTok{, }\StringTok{"red"}\NormalTok{),}
    \DataTypeTok{pch =} \DecValTok{19}\NormalTok{,}
    \DataTypeTok{bg =} \StringTok{"white"}
\NormalTok{  )}
\NormalTok{\}}
\KeywordTok{plot}\NormalTok{(Y_all, }\DataTypeTok{pch =} \DecValTok{19}\NormalTok{, }\DataTypeTok{col =} \StringTok{"grey"}\NormalTok{)}
\KeywordTok{points}\NormalTok{(Ystar, }\DataTypeTok{pch =} \DecValTok{19}\NormalTok{, }\DataTypeTok{col =} \StringTok{"red"}\NormalTok{)}
\KeywordTok{points}\NormalTok{(Y, }\DataTypeTok{pch =} \DecValTok{19}\NormalTok{, }\DataTypeTok{col =} \StringTok{"black"}\NormalTok{)}
\KeywordTok{mylegend}\NormalTok{()}
\end{Highlighting}
\end{Shaded}

\includegraphics{multiple_imputation_files/figure-latex/impute-1.pdf}

Clearly, these initial values are a bad fit to the data. Nevertheless,
draw \(\mu\) and \(\Sigma\) conditioned on this set of imputed values.

\begin{Shaded}
\begin{Highlighting}[]
\NormalTok{mu_star <-}\StringTok{ }\KeywordTok{draw_mu}\NormalTok{(Ystar, Sigma_star, mu_}\DecValTok{0}\NormalTok{, Sigma_}\DecValTok{0}\NormalTok{)}
\NormalTok{mu_star}
\end{Highlighting}
\end{Shaded}

\begin{verbatim}
##           [,1]    [,2]
## [1,] -2.319796 1.05317
\end{verbatim}

\begin{Shaded}
\begin{Highlighting}[]
\NormalTok{Sigma_star <-}\StringTok{ }\KeywordTok{draw_Sigma}\NormalTok{(Ystar, mu_star, v0, S0)}
\NormalTok{Sigma_star}
\end{Highlighting}
\end{Shaded}

\begin{verbatim}
##           [,1]     [,2]
## [1,] 48.234522 6.449814
## [2,]  6.449814 2.684697
\end{verbatim}

Draw a new set of imputed values, conditioned on the current \(\mu^*\)
and \(\Sigma^*\).

\begin{Shaded}
\begin{Highlighting}[]
\NormalTok{Ystar <-}\StringTok{ }\KeywordTok{mvnorm_fill_missing}\NormalTok{(Y, mu_star, Sigma_star)}
\KeywordTok{plot}\NormalTok{(Y_all, }\DataTypeTok{pch =} \DecValTok{19}\NormalTok{, }\DataTypeTok{col =} \StringTok{"grey"}\NormalTok{)}
\KeywordTok{points}\NormalTok{(Ystar, }\DataTypeTok{pch =} \DecValTok{19}\NormalTok{, }\DataTypeTok{col =} \StringTok{"red"}\NormalTok{)}
\KeywordTok{points}\NormalTok{(Y, }\DataTypeTok{pch =} \DecValTok{19}\NormalTok{, }\DataTypeTok{col =} \StringTok{"black"}\NormalTok{)}
\KeywordTok{mylegend}\NormalTok{()}
\end{Highlighting}
\end{Shaded}

\includegraphics{multiple_imputation_files/figure-latex/impute_2-1.pdf}

This is still a bad fit to the data, but, though it is hard to tell, it
has improved slightly. Doing this repeatedly (in a loop) gradually
improves the estimates of the covariance.

\begin{Shaded}
\begin{Highlighting}[]
\KeywordTok{par}\NormalTok{(}\DataTypeTok{mfrow =} \KeywordTok{c}\NormalTok{(}\DecValTok{5}\NormalTok{, }\DecValTok{4}\NormalTok{), }\DataTypeTok{mar =} \KeywordTok{c}\NormalTok{(}\DecValTok{3}\NormalTok{, }\DecValTok{3}\NormalTok{, }\DecValTok{2}\NormalTok{, }\FloatTok{0.1}\NormalTok{), }\DataTypeTok{cex =} \FloatTok{0.4}\NormalTok{)}
\ControlFlowTok{for}\NormalTok{ (i }\ControlFlowTok{in} \DecValTok{1}\OperatorTok{:}\DecValTok{20}\NormalTok{) \{}
\NormalTok{  mu_star <-}\StringTok{ }\KeywordTok{draw_mu}\NormalTok{(Ystar, Sigma_star, mu_}\DecValTok{0}\NormalTok{, Sigma_}\DecValTok{0}\NormalTok{)[}\DecValTok{1}\NormalTok{,]}
\NormalTok{  Sigma_star <-}\StringTok{ }\KeywordTok{draw_Sigma}\NormalTok{(Ystar, mu_star, v0, S0)}
\NormalTok{  Ystar <-}\StringTok{ }\KeywordTok{mvnorm_fill_missing}\NormalTok{(Y, mu_star, Sigma_star)}
\NormalTok{  title <-}\StringTok{ }\KeywordTok{sprintf}\NormalTok{(}\StringTok{"iteration: %d, covariance: %.2f"}\NormalTok{, i, Sigma_star[}\DecValTok{2}\NormalTok{, }\DecValTok{1}\NormalTok{])}
  \KeywordTok{plot}\NormalTok{(Y_all, }\DataTypeTok{pch =} \DecValTok{19}\NormalTok{, }\DataTypeTok{col =} \StringTok{"grey"}\NormalTok{, }\DataTypeTok{xlab =} \StringTok{""}\NormalTok{, }\DataTypeTok{ylab =} \StringTok{""}\NormalTok{, }\DataTypeTok{main =}\NormalTok{ title)}
  \KeywordTok{points}\NormalTok{(Ystar, }\DataTypeTok{pch =} \DecValTok{19}\NormalTok{, }\DataTypeTok{col =} \StringTok{"red"}\NormalTok{)}
  \KeywordTok{points}\NormalTok{(Y, }\DataTypeTok{pch =} \DecValTok{19}\NormalTok{, }\DataTypeTok{col =} \StringTok{"black"}\NormalTok{)}
\NormalTok{\}}
\end{Highlighting}
\end{Shaded}

\includegraphics{multiple_imputation_files/figure-latex/loop-1.pdf}

After a lot of MCMC samples, we can generate a distribution of
covariance values, which provides an uncertainty estimate on the
covariance.

\begin{Shaded}
\begin{Highlighting}[]
\NormalTok{cov_ab <-}\StringTok{ }\KeywordTok{numeric}\NormalTok{(}\DecValTok{1000}\NormalTok{)}
\ControlFlowTok{for}\NormalTok{ (i }\ControlFlowTok{in} \KeywordTok{seq_along}\NormalTok{(cov_ab)) \{}
\NormalTok{  mu_star <-}\StringTok{ }\KeywordTok{draw_mu}\NormalTok{(Ystar, Sigma_star, mu_}\DecValTok{0}\NormalTok{, Sigma_}\DecValTok{0}\NormalTok{)[}\DecValTok{1}\NormalTok{,]}
\NormalTok{  Sigma_star <-}\StringTok{ }\KeywordTok{draw_Sigma}\NormalTok{(Ystar, mu_star, v0, S0)}
\NormalTok{  Ystar <-}\StringTok{ }\KeywordTok{mvnorm_fill_missing}\NormalTok{(Y, mu_star, Sigma_star)}
\NormalTok{  cov_ab[i] <-}\StringTok{ }\NormalTok{Sigma_star[}\DecValTok{2}\NormalTok{, }\DecValTok{1}\NormalTok{]}
\NormalTok{\}}
\KeywordTok{hist}\NormalTok{(cov_ab, }\DataTypeTok{xlab =} \StringTok{"Covariance estimate"}\NormalTok{, }\DataTypeTok{ylab =} \StringTok{"Count"}\NormalTok{, }\DataTypeTok{main =} \StringTok{"Covariance estimate"}\NormalTok{)}
\KeywordTok{abline}\NormalTok{(}\DataTypeTok{v =}\NormalTok{ true_cov[}\DecValTok{2}\NormalTok{, }\DecValTok{1}\NormalTok{], }\DataTypeTok{col =} \StringTok{"red"}\NormalTok{, }\DataTypeTok{lwd =} \DecValTok{2}\NormalTok{)}
\KeywordTok{legend}\NormalTok{(}\StringTok{"topright"}\NormalTok{, }\DataTypeTok{legend =} \StringTok{"True cov."}\NormalTok{, }\DataTypeTok{lty =} \StringTok{"solid"}\NormalTok{, }\DataTypeTok{lwd =} \DecValTok{2}\NormalTok{, }\DataTypeTok{col =} \StringTok{"red"}\NormalTok{)}
\end{Highlighting}
\end{Shaded}

\includegraphics{multiple_imputation_files/figure-latex/cov_estimate-1.pdf}


\end{document}
